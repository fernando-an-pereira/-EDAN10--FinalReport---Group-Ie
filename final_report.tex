\documentclass[a4paper]{article}
\usepackage[T1]{fontenc}
\usepackage[english]{babel}
\usepackage{amsmath}
\usepackage{graphicx}
\usepackage{url}
\usepackage{verbatim}

% Hypertext
\usepackage{hyperref}

%Bibliobraphy
\usepackage{natbib}
\usepackage{bibtopic}
\usepackage{url}

\usepackage[utf8]{inputenc} % Krävs för att svenska tecken ska läsas korrekt i vissa system.
%\usepackage[latin1]{inputenc} % Om svenska tecken inte fungerar korrekt, försök att byta ut föregående rad mot denna (eller testa utan någon av raderna)

%Allow the use of \verbatimtabinput which includes external files, and handling tabs correctly
\usepackage{moreverb}
\def\verbatimtabsize{4\relax} 

\bibliographystyle{unsrt}

%Remove red boxes due to the hyperref
\hypersetup{
    colorlinks,
    citecolor=black,
    filecolor=black,
    linkcolor=black,
    urlcolor=black
}
%%%%%%%%%%%%%%%%%%%%%%%%%%%%%

\title{EDAN10 -- Group I-e\\--\\ Final Report \\
Introduction of SCM in a company 
%maybe add a better title
}

\author{Einar Holst \\
Fernando de Andrade Pereira \\
Hani Fakhouri
}

\begin{document}
\maketitle
\thispagestyle{empty}
\clearpage

\tableofcontents
\thispagestyle{empty}
\clearpage

\setcounter{page}{1}

\section{Keywords}
%introduce one keyword per line
\begin{itemize}
\item Software Configuration Managment Plan
\item Simultaneous update
\item Maintaining releases
\item Versioning tools
\item Configuration Item
\item Change management
\item Change Control Board
\item Introducing Software Configuration Managment
\end{itemize}


\section{Topic}
This paper discusses the introduction of SCM in a company (specifically the \emph{PonteVecchio Software}).

The project can be considered quite relevant due to the increasing productivity of the corporation which it may gain by using a SCM tool.

The main stakeholder of this project is the company management because their projects will be finished faster and as a result more products can be sold. This will also result in giving the company a chance to grow further without being interupted by the issues they are having today.

The developers and testers of the company are identified as secondary stakeholders because they should agree with the changes in the company and are directly effected by these changes. As some of the issues the developers and testers are having are caused by the lack of configuration managment and adding support will reduce the stress level for the developers and testers and will increase the moral of the team, making it a more pleasant work environment.

Other secondary stakeholders are costumers of this company. While not a direct stakeholder we have to consider, that their interests are effected by the work done in this project as it will increase the productivity of the company. Thus costumers will get a higher value back on their money and receive higher quality and lower delivery times.

\section{Problem statement}
A company (called \emph{PonteVecchio Software} or PVS for short) is a small software development company that has grown from 2 to 15 developers in the last 10 years. They have never introduced any form of configuration managment (CM for short) and are starting to notice that they have problems they can not quiet explain. They have heard that CM could help them with some of these problems and have requested a plan to introduce CM to the company in steps that will not disturbe their production but at the same time solve their issues.

Currently the company works on their source code using a single network resource to and from which everyone reads and writes. This is likely to be the cause of one of the issues they have been having which is that code that has been done seems to disappear from time to time without a trace. This is commonly known as the \emph{simultaneous update} problem in CM and is normaly solved using suitable software tools.

The simultaneous update problem that has been creeping into development has meant that the company has been forced to introduce a \emph{Waterfall method} of development that allows them dictate who can edit what files and when in order to minimize the problem. This in some ways might slow down production as delays in a single part of the plan set forward can cause chain reactions that delay the whole project.

Another issue that they have because of the shared network resource is that they do not have the ability to maintain old releases that have been delivered to costumers. While a copy of what was delivered to the costumer is kept on a CD which was created at release. While this copy does exist it does not provide any benefit to the company other then the ability to make a copy of it and deliver it to clients again if it is lost.

There is also an issue with that the company has no clear process for handling change requests which has resulted in some confusion among the employees. Mostly this happens when some interface or class changes how it is intended to be used or methods change name without the information that it will be changed or even has changed reaches all the developers.

While some of these issues could be easily fixed the company managment has also decided that they want to use exclusivly free tools to do so. While not a major problem it is one extra consideration to take into account when solving the issues the company has presently.

\section{Education}
Using a configuration management system whithin a development team is not easy, or very effective, if the whole team has not bought into it[11]. Therefore it is essential when introducing CM into the team to make sure that every member of the team is convinced of using CM and clearly understands the reasons of using it. To achieve that some kind of education needs to be done.

We make sure to explain why CM must be introduced. For example by listing all the software management related problems the team is experiencing (in a board) and then eventually showing how, when applying CM, each of these problems can be tackled and minimized or completely dissapear and be solved. Classical problems such as the simultaneous update and the shared data are good examples to show for the team and explain for it how they can be solved and the benefits of solving them. Furthermore the advantages of using CM must also be illustrated. Advanteages like increased productivity, better communication and coordination between the members, better code management, better change management and ofcourse better controll of the software being developed by the team. Showing some statistics of other teams productivity after introducing CM might be a good way to illustrate the importance of CM usage. Moreover it is important to explain to the team that all these benefits come with an extra work that should be done. That extra work might be seen as “extra” at the beginning but eventually it will become a part of the overall workflow. 

After explaining the general concept of CM, its parts and the advantages of apllying CM a configuration manager should be chosen. Generally, a software configuration manager must have a solid base of training and/or practical experiences in at least one of two areas: software engineering and management[22].  We would suggest one of the developers who might volunteraly want to play this role or one of us who has a solid basic background in the area. Then the responsibilities and scope of authority of the CM manager must be explained. Responsibilities include development and/or administration of the SCM plan, review of the SCM plan, support of the CCB on software configuration control and management issues, providing status accounting reports to intreseted parties, serving as the interface between software engineering and software quality assurance and maintenance of software integrity[22]. All these responsibilities shall be clearly explained and should be fully understood by the person who shall act as the CM manager. 

An important part of the education process is to run an in depth education of the SCM tool the team shall be using. As the team has never used such a tool it will be explained why the tool is needed and how it will improve the overall work process and the code management part. A good way to do that is to show the tool in live via a projector so that the team gets familiar with the tool UI. Different concepts of the SCM tool must also be explained. Concepts such as branching, merging, merge conflicts and different strategies on how to resolve these conflicts. Furthermore, we shall mention the different merging and branching concepts that exist. Because the team is relatively small we would recommend to use an early branching and relaxed merging styles. We would not need to explain all the styles that exist but just mention these so that the team is aware that different styles exist and they would know when to switch to the right strategy. A more concentrated and detailed demostration and explanation shall be applied on the strategy that will be used by the team. For example we shall explain that higher safety risk and less effort would mean a relaxed policy toward codelines and fewer integration. Linse. Less risk on the other hand implies stricter codeline policies, more codelines and more integration effort[1].

The education process shall be conducted for several days as it includes different and important parts which are assential to understand. The place and time of the educational meetings shall be properly chosen so that every member of the team can come and participate.


\section{CM Plan}

\subsection{Status Accounting}

As in any management discipline, an accounting of status and activities is required for configuration management [4] and thus should be included in the SCM plan. Configuration status accounting is a management information system that provides for tracebility and status information within the CM process. It is a system of reports and records documenting CIs and baselines throughout the product lifecycle[4]. 

It is important to construct an information gathering mechanism, i.e. how the status accounting information is gathered. We would suggest that the different information should be entered into the CM database by the initiators of the SCM activities rather than by the CM manager, who if doing that, will end up chasing the activities and updating the status accounting data. The CM manager on the other hand will have an overall view of the information in the database and might request different updates from the initiators rather than chasing them. In order to make the above process as easy and as automated as possible well structured reports should be used by the team. These reports should be made so that they are easily traced. Such a report shall include date and time, unique and significant report number (version), title, initiator, status, priority, description, specification, CI location in the system and manager. We would ferthermore divide there reports into three categories which define in what context each report is created which are control, identification and documentation.

The identification report inclde:
\begin{itemize}
  \item CI name and number
  \item Status
  \item Initiator (developer)
  \item Requirement reference
  \item Specification reference
  \item Parentage (part of another document)
  \item CI manager
\end{itemize}

The control report include:
\begin{itemize}
  \item Change number
  \item Change title
  \item Change class
  \item Priority
  \item Submission date
  \item Implementation date
  \item Verification date
  \item Status
\end{itemize}

\subsection{Auditing}

By conductuing configuration audits the project verifies that the technical documentation matches and reflects product performance. Auditing is as essential part of CM and thus should also be included in the CM plan.

It is important to diffirentiate between two types of audits, namely functional and physical configuration audits. Functional audits verify the functional characteristics while the physical audits verify th form and fir of the product. We would suggest to conduct functional audits in conjuction with integration testing as test data becomes available. During this functions audit, the CM manager shall review baseline documents, trace the product to its initial specifiaction, review status change and problem reports and compare test results to those of the requirements review[4]. 

 A mechanism on how to conduct an audit shall address the requirement (what is to be accomplished?), the preparation for the audit, coduct and followup process. Here checklists, which provide a script and ensure that nothing is left out, shall be used. We would suggest a simple checklist with a Yes No options for each data. Preparation process includes the establishment of the exact target for the audit and the definition of the data package (what documentation needs to be reviewed). This data package should be made available to all who will conduct the audit. The audit team would typically include a chairperson, a recorder, the CM manager and other members of the team selected on the basis of their special interests or knowledge. When all the above is ready the audit team meeting is scheduled and conducted. The data package is reviewed, assignments are made and procedures and checklists are also reviewed. The audit is then formally scheduled and announced, previously coordinated with the party to be audited. A final briefing with a review of responsibilities and checklists may be helpful. The audit begins with a meeting and the meeting is conducted by the chairperson. During the meeting an overview of the audit is given, the schedule and scenario is announced. Upon completion of the audit, the audit team meets seperately to discuss findings in order to prepare a report. Then official problem resolution action items are established. The most important part of the auditing process is the followup part[4]. When the final audit meeting is established, action items and report of findings should be published and distributed to all participants and to the management. When these items and findings are completed and resolved, the audit team reviews and verifies that the problems were actually resolved. Finally when all actions are closed, a final report is to be prepared. Necessary approvals and signatures need to be obtained. Then distribution and archiving is followed. For complete closure of the audit process it is important that the CM manager ensures that the configuration status accounting database has been updated with all required audit information.

\section{SCM Tools}

\section{Conclusion}

\newpage
\appendix
\section{Literature}

The literature that will be used in this project is prioritized as primary -- most relevant to this work -- and secondary -- less relevant, but also important. 
\begin{btSect}[alpha]{primary}
\subsection{Primary}
\btPrintAll
\end{btSect}

\begin{btSect}[alpha]{secondary}
\subsection{Secondary}
\btPrintAll
\end{btSect}

\end{document}
